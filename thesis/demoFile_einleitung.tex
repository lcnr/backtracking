\chapter{Einleitung}\label{einleitung}
Viele Probleme lassen sich als eine Sequenz $x_{1}, x_{2}, x_{3} \dots x_{n}$ für 
welche die Bedingung $P_{n}(x_{1}, x_{2}, x_{3} \dots x_{n})$ gelten soll modelieren.
Damit Backtracking zu deren Lösung eingesetzt werden kann müssen außerdem noch
Untereigenschaften $P_{v}(x_{1}, x_{2}, x_{3} \dots x_{v})$ für alle $v \in [ \, 0, n) \,$ 
mit folgenden Eigenschaften existieren:
\begin{enumerate}
  \item $P_{0}()$ gilt immer
  \item $P_{v + 1}(x_{1}, x_{2}, x_{3} \dots x_{v + 1})$ gilt nur, wenn $P_{v}(x_{1}, x_{2}, x_{3} \dots x_{v})$ gilt
  \item wenn $P_{v}(x_{1}, x_{2}, x_{3} \dots x_{v})$ gilt, ist $P_{v + 1}(x_{1}, x_{2}, x_{3} \dots x_{v+1})$ einfach zu testen
\end{enumerate}