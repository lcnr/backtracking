\chapter{Einleitung}\label{einleitung}
Hier beginnt die eigentliche Arbeit.

Text text text.

In der Ecke eines Zimmers stand ein Schwert \cite{btt:1998}. Die helle, stählerne Fläche seiner
Klinge erglänzte, vom Strahle der Sonne berührt, in rötlichem Scheine. Stolz
hielt das Schwert Umschau im Zimmer; es sah, daß alles sich an seinem Glasten
weidete. Alles? - Nicht doch! Dort auf dem Tische lag, müßig an ein Tintenfaß
gelehnt, eine Feder, der es nicht im mindesten einfiel sich vor der
glitzernden Majestät jener Waffe zu beugen. - Das ergrimmte das Schwert und es
begann also zu sprechen: »Wer bist du wohl, nichtswürdig Ding, daß du nicht
gleich den andern vor meinem Glanz dich beugst und ihn bewunderst? Sieh nur
um dich! Alle Geräte stehen ehrfurchtsvoll in tiefes Dunkel gehüllt. Mich
allein, mich hat die helle beglückende Sonne zu ihrem Liebling erkoren; sie
belebt mich mit ihrem wonnigen Flammenkusse, und ich lohne ihrs, indem ich ihr
Licht tausendfach widerstrahle. Mächtigen Fürsten nur ziemt es, in leuchtendem
Gewande daherzuschreiten. Die Sonne kennt meine Macht, darum legt sie mir den
königlichen Purpur ihrer Strahlen um die Schultern.«

In der Ecke eines Zimmers stand ein Schwert. Die helle, stählerne Fläche seiner
Klinge erglänzte, vom Strahle der Sonne berührt, in rötlichem Scheine. Stolz
hielt das Schwert Umschau im Zimmer; es sah, daß alles sich an seinem Glasten
weidete. Alles? - Nicht doch! Dort auf dem Tische lag, müßig an ein Tintenfaß
gelehnt, eine Feder, der es nicht im mindesten einfiel sich vor der
glitzernden Majestät jener Waffe zu beugen. - Das ergrimmte das Schwert und es
begann also zu sprechen: »Wer bist du wohl, nichtswürdig Ding, daß du nicht
gleich den andern vor meinem Glanz dich beugst und ihn bewunderst? Sieh nur
um dich! Alle Geräte stehen ehrfurchtsvoll in tiefes Dunkel gehüllt. Mich
allein, mich hat die helle beglückende Sonne zu ihrem Liebling erkoren; sie
belebt mich mit ihrem wonnigen Flammenkusse, und ich lohne ihrs, indem ich ihr
Licht tausendfach widerstrahle. Mächtigen Fürsten nur ziemt es, in leuchtendem
Gewande daherzuschreiten. Die Sonne kennt meine Macht, darum legt sie mir den
königlichen Purpur ihrer Strahlen um die Schultern.«

In der Ecke eines Zimmers stand ein Schwert. Die helle, stählerne Fläche seiner
Klinge erglänzte, vom Strahle der Sonne berührt, in rötlichem Scheine. Stolz
hielt das Schwert Umschau im Zimmer; es sah, daß alles sich an seinem Glasten
weidete. Alles? - Nicht doch! Dort auf dem Tische \cite{btt:1998} lag, müßig an ein Tintenfaß
gelehnt, eine Feder, der es nicht im mindesten einfiel sich vor der
glitzernden Majestät jener Waffe zu beugen. - Das ergrimmte das Schwert und es
begann also zu sprechen: »Wer bist du wohl, nichtswürdig Ding, daß du nicht
gleich den andern vor meinem Glanz dich beugst und ihn bewunderst? Sieh nur
um dich! Alle Geräte stehen ehrfurchtsvoll in tiefes Dunkel gehüllt. Mich
allein, mich hat die helle beglückende Sonne zu ihrem Liebling erkoren; sie
belebt mich mit ihrem wonnigen Flammenkusse, und ich lohne ihrs, indem ich ihr
Licht tausendfach widerstrahle. Mächtigen Fürsten nur ziemt es, in leuchtendem
Gewande daherzuschreiten. Die Sonne kennt meine Macht, darum legt sie mir den
königlichen Purpur ihrer Strahlen um die Schultern.«

In der Ecke eines Zimmers stand ein Schwert. Die helle, stählerne Fläche seiner
Klinge erglänzte, vom Strahle der Sonne berührt, in rötlichem Scheine. Stolz
hielt das Schwert Umschau im Zimmer; es sah, daß alles sich an seinem Glasten
weidete. Alles? - Nicht doch! Dort auf dem Tische lag, müßig an ein Tintenfaß
gelehnt, eine Feder, der es nicht im mindesten einfiel sich vor der
glitzernden Majestät jener Waffe zu beugen. - Das ergrimmte das Schwert und es
begann also zu sprechen: »Wer bist du wohl, nichtswürdig Ding, daß du nicht
gleich den andern vor meinem Glanz dich beugst und ihn bewunderst? Sieh nur
um dich! Alle Geräte stehen ehrfurchtsvoll in tiefes Dunkel gehüllt. Mich
allein, mich hat die helle beglückende Sonne zu ihrem Liebling erkoren; sie
belebt mich mit ihrem wonnigen Flammenkusse, und ich lohne ihrs, indem ich ihr
Licht tausendfach widerstrahle. Mächtigen Fürsten nur ziemt es, in leuchtendem
Gewande daherzuschreiten. Die Sonne kennt meine Macht, darum legt sie mir den
königlichen Purpur ihrer Strahlen um die Schultern.«

In der Ecke eines Zimmers stand ein Schwert. Die helle, stählerne Fläche seiner
Klinge erglänzte, vom Strahle der Sonne berührt, in rötlichem Scheine. Stolz
hielt das Schwert Umschau im Zimmer; es sah, daß alles sich an seinem Glasten
weidete. Alles? - Nicht doch! Dort auf dem Tische lag, müßig an ein Tintenfaß
gelehnt, eine Feder, der es nicht im mindesten einfiel sich vor der
glitzernden Majestät jener Waffe zu beugen. - Das ergrimmte das Schwert und es
begann also zu sprechen: »Wer bist du wohl, nichtswürdig Ding, daß du nicht
gleich den andern vor meinem Glanz dich beugst und ihn bewunderst? Sieh nur
um dich! Alle Geräte stehen ehrfurchtsvoll in tiefes Dunkel gehüllt. Mich
allein, mich hat die helle beglückende Sonne zu ihrem Liebling erkoren; sie
belebt mich mit ihrem wonnigen Flammenkusse, und ich lohne ihrs, indem ich ihr
Licht tausendfach widerstrahle. Mächtigen Fürsten nur ziemt es, in leuchtendem
Gewande daherzuschreiten. Die Sonne kennt meine Macht, darum legt sie mir den
königlichen Purpur ihrer Strahlen um die Schultern.«

In der Ecke eines Zimmers stand ein Schwert. Die helle, stählerne Fläche seiner
Klinge erglänzte, vom Strahle der Sonne berührt, in rötlichem Scheine. Stolz
hielt das Schwert Umschau im Zimmer; es sah, daß alles sich an seinem Glasten
weidete. Alles? - Nicht doch! Dort auf dem Tische lag, müßig an ein Tintenfaß
gelehnt, eine Feder, der es nicht im mindesten einfiel sich vor der
glitzernden Majestät jener Waffe zu beugen. - Das ergrimmte das Schwert und es
begann also zu sprechen: »Wer bist du wohl, nichtswürdig Ding, daß du nicht
gleich den andern vor meinem Glanz dich beugst und ihn bewunderst? Sieh nur
um dich! Alle Geräte stehen ehrfurchtsvoll in tiefes Dunkel gehüllt. Mich
allein, mich hat die helle beglückende Sonne zu ihrem Liebling erkoren; sie
belebt mich mit ihrem wonnigen Flammenkusse, und ich lohne ihrs, indem ich ihr
Licht tausendfach widerstrahle. Mächtigen Fürsten nur ziemt es, in leuchtendem
Gewande daherzuschreiten. Die Sonne kennt meine Macht, darum legt sie mir den
königlichen Purpur ihrer Strahlen um die Schultern.«

In der Ecke eines Zimmers stand ein Schwert. Die helle, stählerne Fläche seiner
Klinge erglänzte, vom Strahle der Sonne berührt, in rötlichem Scheine. Stolz
hielt das Schwert Umschau im Zimmer; es sah, daß alles sich an seinem Glasten
weidete. Alles? - Nicht doch! Dort auf dem Tische lag, müßig an ein Tintenfaß
gelehnt, eine Feder, der es nicht im mindesten einfiel sich vor der
glitzernden Majestät jener Waffe zu beugen. - Das ergrimmte das Schwert und es
begann also zu sprechen: »Wer bist du wohl, nichtswürdig Ding, daß du nicht
gleich den andern vor meinem Glanz dich beugst und ihn bewunderst? Sieh nur
um dich! Alle Geräte stehen ehrfurchtsvoll in tiefes Dunkel gehüllt. Mich
allein, mich hat die helle beglückende Sonne zu ihrem Liebling erkoren; sie
belebt mich mit ihrem wonnigen Flammenkusse, und ich lohne ihrs, indem ich ihr
Licht tausendfach widerstrahle. Mächtigen Fürsten nur ziemt es, in leuchtendem
Gewande daherzuschreiten. Die Sonne kennt meine Macht, darum legt sie mir den
königlichen Purpur ihrer Strahlen um die Schultern.«

In der Ecke eines Zimmers stand ein Schwert. Die helle, stählerne Fläche seiner
Klinge erglänzte, vom Strahle der Sonne berührt, in rötlichem Scheine. Stolz
hielt das Schwert Umschau im Zimmer; es sah, daß alles sich an seinem Glasten
weidete. Alles? - Nicht doch! Dort auf dem Tische lag, müßig an ein Tintenfaß
gelehnt, eine Feder, der es nicht im mindesten einfiel sich vor der
glitzernden Majestät jener Waffe zu beugen. - Das ergrimmte das Schwert und es
begann also zu sprechen: »Wer bist du wohl, nichtswürdig Ding, daß du nicht
gleich den andern vor meinem Glanz dich beugst und ihn bewunderst? Sieh nur
um dich! Alle Geräte stehen ehrfurchtsvoll in tiefes Dunkel gehüllt. Mich
allein, mich hat die helle beglückende Sonne zu ihrem Liebling erkoren; sie
belebt mich mit ihrem wonnigen Flammenkusse, und ich lohne ihrs, indem ich ihr
Licht tausendfach widerstrahle. Mächtigen Fürsten nur ziemt es, in leuchtendem
Gewande daherzuschreiten. Die Sonne kennt meine Macht, darum legt sie mir den
königlichen Purpur ihrer Strahlen um die Schultern.«

In der Ecke eines Zimmers stand ein Schwert. Die helle, stählerne Fläche seiner
Klinge erglänzte, vom Strahle der Sonne berührt, in rötlichem Scheine. Stolz
hielt das Schwert Umschau im Zimmer; es sah, daß alles sich an seinem Glasten
weidete. Alles? - Nicht doch! Dort auf dem Tische lag, müßig an ein Tintenfaß \cite{btt:1998}
gelehnt, eine Feder, der es nicht im mindesten einfiel sich vor der
glitzernden Majestät jener Waffe zu beugen. - Das ergrimmte das Schwert und es
begann also zu sprechen: »Wer bist du wohl, nichtswürdig Ding, daß du nicht
gleich den andern vor meinem Glanz dich beugst und ihn bewunderst? Sieh nur
um dich! Alle Geräte stehen ehrfurchtsvoll in tiefes Dunkel gehüllt (s.\ Bild~\ref{fig:bild1}). Mich
allein, mich hat die helle beglückende Sonne zu ihrem Liebling erkoren; sie
belebt mich mit ihrem wonnigen Flammenkusse, und ich lohne ihrs, indem ich ihr
Licht tausendfach widerstrahle. Mächtigen Fürsten nur ziemt es, in leuchtendem
Gewande daherzuschreiten. Die Sonne kennt meine Macht, darum legt sie mir den
königlichen Purpur ihrer Strahlen um die Schultern.«

\begin{figure}[t]\centering
  \includegraphics[width = 120mm]{bilder/DemoFile_tiger}
  \caption{Der {\textbf Tiger} ({\em Panthera tigris}) ist eine in Asien
           verbreitete, wegen der schwarzen Streifung auf goldgelbem bis
	   rotbraunem Grund unverkennbare Katze. Er ist die größte und
	   stärkste aller Raubkatzen.}\label{fig:bild1}
\end{figure}

In der Ecke eines Zimmers stand ein Schwert. Die helle, stählerne Fläche seiner
Klinge erglänzte, vom Strahle der Sonne berührt, in rötlichem Scheine. Stolz
hielt das Schwert Umschau im Zimmer; es sah, daß alles sich an seinem Glasten
weidete. Alles? - Nicht doch! Dort auf dem Tische lag, müßig an ein Tintenfaß
gelehnt, eine Feder, der es nicht im mindesten einfiel sich vor der
glitzernden Majestät jener Waffe zu beugen. - Das ergrimmte das Schwert und es
begann also zu sprechen: »Wer bist du wohl, nichtswürdig Ding, daß du nicht
gleich den andern vor meinem Glanz dich beugst und ihn bewunderst? Sieh nur
um dich! Alle Geräte stehen ehrfurchtsvoll in tiefes Dunkel gehüllt. Mich
allein, mich hat die helle beglückende Sonne zu ihrem Liebling erkoren; sie
belebt mich mit ihrem wonnigen Flammenkusse, und ich lohne ihrs, indem ich ihr
Licht tausendfach widerstrahle. Mächtigen Fürsten nur ziemt es, in leuchtendem
Gewande daherzuschreiten. Die Sonne kennt meine Macht, darum legt sie mir den
königlichen Purpur ihrer Strahlen um die Schultern.«


