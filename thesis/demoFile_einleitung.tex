\chapter{Einleitung}\label{einleitung}
Es ist oft möglich Probleme als eine Sequenz $x_{1}, x_{2}, x_{3} \dots x_{n}$ für 
welche die Bedingung $P_{n}(x_{1}, x_{2}, x_{3} \dots x_{n})$ gelten soll darzustellen.
Damit Backtracking zu deren Lösung eingesetzt werden kann müssen außerdem noch
Untereigenschaften $P_{v}(x_{1}, x_{2}, x_{3} \dots x_{v})$ für alle $v \in [ \, 0, n) \,$ 
mit folgenden Eigenschaften existieren:
\begin{enumerate}
  \item $P_{0}()$ gilt immer
  \item $P_{v + 1}(x_{1}, x_{2}, x_{3} \dots x_{v + 1})$ gilt nur, wenn $P_{v}(x_{1}, x_{2}, x_{3} \dots x_{v})$ gilt
  \item wenn $P_{v}(x_{1}, x_{2}, x_{3} \dots x_{v})$ gilt, ist $P_{v + 1}(x_{1}, x_{2}, x_{3} \dots x_{v+1})$ einfach zu testen
\end{enumerate}

Somit kann man alle Sequenzen $x_{1}, x_{2}, x_{3} \dots x_{v} \dots x_{q}$ mit $q > v$ ignorieren,
falls $P_{v}(x_{1}, x_{2}, x_{3} \dots x_{v})$ nicht gilt.
Der darauf basierende Backtracking Algorithmus(\textbf{B}) kann nun wie folgt implementiert werden:
\begin{minted}[linenos, fontsize=\small]{rust}
pub fn b<T: Sequence>(initial: T, n: usize) -> Vec<T> {
    if !initial.satisfies_condition() {
        return Vec::new();
    } else if n == 0 {
        return vec![initial];
    }

    // all sequences of length n which satisfy the condition
    let mut results = Vec::new();

    // the current sequence, starts with just the initial state
    let mut states = Vec::new();

    let steps = initial.next_steps().into_iter();
    states.push((initial, steps));

    // run while there is still a state with an unchecked possible next step
    while let Some((state, steps)) = states.last_mut() {
        // take the next unchecked possible step of the current state,
        // in case there are no unchecked steps left, simply discard the current state
        // as all possible sequences have already been tried.
        if let Some(step) = steps.next() {
            // compute the result of this step
            let next_state = state.apply_step(step);
            // does this new state still satisfy the condition,
            // if not we can simply discard it
            if next_state.satisfies_condition() {
                // if the sequence is already n elements long,
                // it is correct and can be added to results.
                // Otherwise we push it onto the stack.
                if states.len() < n {
                    let next_steps = next_state.next_steps().into_iter();
                    states.push((next_state, next_steps));
                } else {
                    results.push(next_state);
                }
            }
        } else {
            states.pop();
        }
    }

    results
}
\end{minted}
Um ein Problem mit diesem Algorithmus lösen zu können, benötigt man einen Datentyp welcher den momentanen Zustand
der Sequenz speichern kann und das folgende Interface implementiert:
\begin{minted}[linenos, fontsize=\small]{rust}
/// A required set of methods needed for the generic backtracking algorithms.
pub trait Sequence {
    type Step;
    type Steps: IntoIterator<Item = Self::Step>;

    /// Checks if this sequence satisfies its condition.
    ///
    /// This function can assume that the  parent of `self` satisfied this condition.
    fn satisfies_condition(&self) -> bool;

    /// generates all possible next steps at this current state.
    fn next_steps(&self) -> Self::Steps;

    /// applies a `step` to `self`, returning the resulting sequence.
    ///
    /// this function will only be called if `self.satisfies_condition() == true`.
    fn apply_step(&self, step: Self::Step) -> Self;
}
\end{minted}