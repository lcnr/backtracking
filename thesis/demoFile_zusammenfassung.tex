\chapter*{Zusammenfassung}

Viele komplexe Probleme der Informatik, zum Beispiel Wegsuche, Schachengines und 
das Ausführen logischer Programme \cite[p. ~19]{DBLP:journals/jlp/SomogyiHC96}, lassen sich durch Backtracking, auch Tiefensuche genannt, lösen.
Basierend auf Donald Knuths "`The Art of Computer Programming"'\cite{TAOCP} wird im folgenden diese Vorgehensweise 
vorgestellt und für einige Probleme Lösunsansätze implementiert. Alle hier verwendeten Programmschnipsel wurden in \textbf{Rust} geschrieben und sind
gut dokumentiert auf \textbf{github}\cite{Kauschke} zu finden.
